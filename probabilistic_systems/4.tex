\documentclass{article}
\usepackage{amsmath}
\usepackage{CJK}
\begin{CJK}{UTF8}{min}

\author{}
  \title{確率システム制御特論 第4回 演習問題}
  \date{\today}

\begin{document}
  \maketitle

\section{問題 1}
問題:ベイズの定理の証明をせよ\\ \\
既知:\\
条件付き確率の定義によって:\\
$$ P(A,B)=P(B|A)P(A)=P(A|B)P(B) \eqno{(1-1)}$$
式$(1-1)$が以下のように変形できます:\\
$$ P(A|B)=\dfrac{P(B|A)P(A)}{P(B)} \eqno{(1-2)}$$
式$(1-2)$はベイズの定理の公式なので、証明完了です\\

\section{問題 2}
\begin{equation} e^{-(x_\alpha-\mu)^2/2\sigma^2} \end{equation}
\begin{equation} e^{\sum_{\alpha=1}^{N}-(x_\alpha-\mu)^2/2\sigma^2} \end{equation}
\begin{equation} (x_\alpha-\mu)^2 \end{equation}
\begin{equation} -\dfrac{1}{2\sigma^4}\sum_{\alpha=1}^{N}(x_\alpha-\mu)^2+\dfrac{N}{2\sigma^2} \end{equation}
\begin{equation} \dfrac{1}{N}\sum_{\alpha=1}^{N}x_\alpha \end{equation}
\begin{equation} \dfrac{1}{N}\sum_{\alpha=1}^{N}(x_\alpha-\bar{\mu})^2 \end{equation}

\end{CJK}
\end{document}
