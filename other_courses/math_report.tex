\documentclass[twocolumn]{article}
\usepackage{amsmath}
\usepackage{CJK}
\usepackage{setspace}
\usepackage{anysize}
\marginsize{2cm}{2cm}{2cm}{2cm}

\begin{CJK}{UTF8}{min}

\author{}
  \title{計算数学特論 第1回 レポート}
  \date{\today}

\begin{document}
  \maketitle

\section{課題1} フェルマーの小定理

\subsection{}
pを素数とする。\\
\begin{spacing}{1.5}
$(Zp^*,\times)$が群の構造を持つことを示せ。\\
($ax+py=gcd(a,p)$を満たすxとyの存在は使用可)\\
\hrule
~\\
証明:\\
(1)閉じている\\
(2)結合法則適用できる:$a\times (b\times c)=(a\times b)\times c$\\
(3)単位元、逆元が存在する\\

(1):\\
$a,b\in Z_p^*$, \ よって, \ $a\times b(modp)\in Z_p^*$\\
$a\times b(modp)=0$とすると\\
$a\times b=kp$\\
よって$\dfrac{a\times b}{p}=k$(kは整数)\\
pは素数とすると、kは整数の可能性がない\\
よって$a\times b(modp)\neq 0$\\

(2):\\
$a,b,c\in Z_p^*$\\
$a\times (b\times c)=(a\times b)\times c$が成り立つ\\

(3):\\
$Z_p^*={1,2,...,p-1}$\\
だから1を含むaが単位元がある\\
$a\in Z_p^*$、$ax+py=gcd(a,p)$を成立できるx,yには:\\
$ax+py=1$\\
$(ax+py)(modp)=1(modp)$\\
$ax(modp)+py(modp)=1$\\
よって:$ax(modp)=1$\\
となる$x=a^-1$が成り立つ、逆元はある\\

(1)(2)(3)によって、$(Zp^*,\times)$が群の構造を持

\subsection{}
1-1を用いて、フェルマーの小定理\\
$a^{p-1}(modp)=1$(aはZp*の任意の要素)\\
を示せ。\\
\hrule
~\\
{1,2,3...,p-1}を{a,2a,3a,...(p-1a)}と見る\\
よって: \ $1a\times 2a\times ...\times (p-1)a\equiv 1\times 2\times ...\times (p-1)(modp)$\\
ここで:$1\times 2\times ...\times (p-1)$の逆元を上の式に掛けると:\\
$a^{p-1}=1(modp)$\\

\end{spacing}

\section{課題2}
\subsection{}
\begin{spacing}{1.5}

p=7, q=13で公開鍵eと秘密鍵dを設定せよ。\\
\hrule
~\\
$n=p\times q=7\times 13=91$\\
$\lambda=lcm(p-1,q-1)=lcm(6,12)=12$\\
$gcd(d,\lambda)=1, d=5,7,11$\\
ここで$d=5$を取る:\\
$ed\equiv 1(mod\lambda), e=5$\\
したがって、$e=5, d=5$\\

\subsection{}

コーディングを以下とする: \\
$a=1,b=2,...,z=26, A=27,B=28,...,Z=52, 0=60,1=61,...,9=69$ \\
(A) kit を送受信せよ。 \\
(B) 各人のイニシャル2文字を送受信せよ。 \\
(C) 学生番号の下4桁を送受信せよ。 \\
\hrule
~\\
(A):\\
送信:\\
$k=11, i=9, t=20, e=5, n=91, q=5$\\
$x=11,9,20$\\
$x_k=11:\ c_k=x^e(modn)=72$\\
$x_i=9:\ c_i=x^e(modn)=81$\\
$x_t=20:\ c_t=x^e(modn)=76$\\
したがって、$c=72,81,76$\\
受信:\\
$c_k=72:\ x_k=c^d(modn)=11$\\
$c_i=81:\ x_i=c^d(modn)=9$\\
$c_t=76:\ x_t=c^d(modn)=20$\\
したがって、$x=11,9,20$\\
\\
(B):\\
名前は"Buwei SHEN"なので、イニシャル2文字は"BS"\\
送信:\\
$B=28, S=45$\\
$x=28,45$\\
$x_B=28:\ c_k=x^e(modn)=84$\\
$x_S=45:\ c_i=x^e(modn)=54$\\
したがって、$c=84,54$\\
受信:\\
$c_B=84:\ x_B=c^d(modn)=28$\\
$c_S=54:\ x_S=c^d(modn)=45$\\
したがって、$x=28,45$\\
\\
(C):\\
学籍番号は"16344229"なので、下4桁は"4229"\\
送信:\\
$"4"=64, "2"=62, "9"=69$\\
$x=64,62,62,69$\\
$x_4=64:\ c_4=x^e(modn)=64$\\
$x_2=62:\ c_2=x^e(modn)=69$\\
$x_2=62:\ c_2=x^e(modn)=69$\\
$x_9=69:\ c_9=x^e(modn)=62$\\
したがって、$c=64,69,69,62$\\
受信:\\
$c_4=64:\ x_4=c^d(modn)=64$\\
$c_2=69:\ x_2=c^d(modn)=62$\\
$c_2=69:\ x_2=c^d(modn)=62$\\
$c_9=62:\ x_9=c^d(modn)=69$\\
したがって、$x=64,62,62,69$\\
\end{spacing}

\end{CJK}
\end{document}
