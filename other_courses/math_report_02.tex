\documentclass{article}
\usepackage{amsmath}
\usepackage{CJK}
\usepackage{setspace}
\usepackage{anysize}
\marginsize{2cm}{2cm}{2cm}{2cm}

\begin{CJK}{UTF8}{min}

\author{}
  \title{計算数学特論 第2回 レポート}
  \date{\today}

\begin{document}
  \maketitle

\begin{spacing}{1.5}
\section{課題1}エルガーマル\\
p=23の場合に原子元aを設定し、xとkを与え、送信文字5がどのように暗号化され、どのように複合化されるかシミュレーションせよ。\\
\hrule
~\\
\textcircled{1} 原始元を探す:\\
$p-1=22$\\
22の約数は2と11だけ\\
$[1,22]$のなか、$a^2$と$a^{11}(modq)\neq 1$の数は$5,7,...,etc$\\
ここで$a=5$を取る\\

\begin{table}[!hbp]
\begin{tabular}{|c|c|c|c|c|}
\hline
  &   &   &   & \\
$5^1(modp)=5$ & $5^2(modp)=2$ & $5^3(modp)=10$ & $5^4(modp)=4$ & $5^5(modp)=20$ \\
  &   &   &   & \\
\hline
  &   &   &   & \\
$5^6(modp)=8$ & $5^7(modp)=17$ & $5^8(modp)=16$ & $5^9(modp)=11$ & $5^{10}(modp)=9$ \\
  &   &   &   & \\
\hline
  &   &   &   & \\
$5^{11}(modp)=22$ & $5^{12}(modp)=18$ & $5^{13}(modp)=21$ & $5^{14}(modp)=13$ & $5^{15}(modp)=19$ \\
  &   &  &  & \\
\hline
  &   &   &   & \\
$5^{16}(modp)=3$ & $5^{17}(modp)=15$ & $5^{18}(modp)=6$ & $5^{19}(modp)=7$ & $5^{20}(modp)=12$ \\
  &   &   &   & \\
\hline
  &   &   &   & \\
$5^{21}(modp)=14$ & $5^{22}(modp)=1$ &   \  &   \  &   \  \\
  &   &   &   & \\
\hline
\end{tabular}
\end{table}
~\\
\textcircled{2} 秘密鍵と公開鍵の計算\\
秘密鍵:$x\in (1,p-1)$, をランダム選択できる、ここで3を選びます\\
公開鍵:$g=5$, \ $p=23$, \ $y=g^x(modp)=5^3(mod23)=10$\\
\textcircled{3} 暗号化\\
送信文字$M=5$\\
$k\in (1,p-1)$, をランダム選択できる、ここで7を選びます\\
$C_1=g^k(modp)=5^7(mod23)=17$\\
$C_2=My^k(modp)=(5\times 10^7)mod23=1$\\
したがって、暗号文$(C_1,C_2)=(17,1)$
\textcircled{3} 復号化\\
$M=C_2C_1^{-x}(modp)=C_2(C_1^x$の逆元$)(modp)$
$=1\times 17^19(mod23)=5$

\section{課題2}\\
(1) p=23の原始元を全部求めよ。\\
(2) 原子元α、乱数a、bを与えて共有鍵を持てることを確認せよ。(3通りの場合を考えること。)\\
\hrule
~\\
\textcircled{1} p=23の原始元を全部求めよ。\\
\begin{table}[!hbp]
\begin{tabular}{|c|c|c|c|c|c|}
\hline
  &   &   &   &  & \\
$2^2(modp)=4$ & $3^2(modp)=9$ & $4^2(modp)=16$ & $5^2(modp)=2$ & $6^2(modp)=13$ & $7^2(modp)=3$ \\
  &   &   &   &  & \\
\hline
  &   &   &   &  & \\
$2^{11}(modp)=1$ & $3^{11}(modp)=1$ & $4^{11}(modp)=1$ & $5^{11}(modp)=22$ & $6^{11}(modp)=1$ & $7^{11}(modp)=22$ \\
  &   &   &   &  & \\
\hline
\hline
  &   &   &   &  & \\
$8^2(modp)=18$  & $9^2(modp)=12$ & $10^2(modp)=8$ & $11^2(modp)=6$ & $12^2(modp)=6$ & $13^2(modp)=8$\\
 &   &   &   &  & \\
\hline
 &   &   &   &  & \\
$8^{11}(modp)=1$  & $9^{11}(modp)=1$ & $10^{11}(modp)=22$ & $11^{11}(modp)=22$ & $12^{11}(modp)=1$ & $13^{11}(modp)=1$\\
 &   &   &   &  & \\
\hline
\hline
 &   &   &   &  & \\
$14^2(modp)=12$  & $15^2(modp)=18$ & $16^2(modp)=3$ & $17^2(modp)=13$ & $18^2(modp)=2$ & $19^2(modp)=16$\\
 &   &   &   &  & \\
\hline
 &   &   &   &  & \\
$14^{11}(modp)=22$  & $15^{11}(modp)=22$ & $16^{11}(modp)=1$ & $17^{11}(modp)=22$ & $18^{11}(modp)=1$ & $19^{11}(modp)=22$\\
 &   &   &   &  & \\
\hline
\hline
 &   &   &   &  & \\
$20^{11}(modp)=9$  & $21^{11}(modp)=4$ & $22^{11}(modp)=1$ &  &  & \\
&   &   &   &  & \\
\hline
&   &   &   &  & \\
$20^{11}(modp)=22$  & $21^{11}(modp)=22$ & $22^{11}(modp)=22$ &  &  & \\
&   &   &   &  & \\
\hline
\end{tabular}
\end{table}
~\\
よって、$p=23$の原始元は${5,7,10,11,14,15,17,19,20,21}$\\
~\\
\textcircled{2} 原子元α、乱数a、bを与えて共有鍵を持てることを確認せよ。(3通りの場合を考えること。)\\
$a=3, b=9$
原子元αを5,7,11を取る\\ \\
$\alpha$が$5$を取る時:\\
$\alpha^a(modp)=5^3(mod23)=10, \alpha^b(modp)=5^9(mod23)=11$\\
共有鍵:$(\alpha^a)^b(modp)=(\alpha^b)^a(modp)=10^9(mod23)=11^3(mod23)=20$\\ \\
$\alpha$が$7$を取る時:\\
$\alpha^a(modp)=7^3(mod23)=21, \alpha^b(modp)=7^9(mod23)=15$\\
共有鍵:$(\alpha^a)^b(modp)=(\alpha^b)^a(modp)=21^9(mod23)=15^3(mod23)=17$\\ \\
$\alpha$が$11$を取る時:\\
$\alpha^a(modp)=5^3(mod23)=20, \alpha^b(modp)=5^9(mod23)=19$\\
共有鍵:$(\alpha^a)^b(modp)=(\alpha^b)^a(modp)=20^9(mod23)=19^3(mod23)=5$\\ \\




\end{spacing}



\end{CJK}
\end{document}
